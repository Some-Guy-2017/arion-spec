% Based on the template from Jean-Philippe Eisenbarth
% https://github.com/jpeisenbarth/SRS-Tex

\documentclass{scrreprt}
\title{Arion Software Requirements Specification}
\author{Joseph Hare}
\date{\today}

\usepackage{xcolor}
\usepackage[shortlabels]{enumitem}
\usepackage[super]{nth}
%\usepackage[utf8]{inputenc}
\usepackage[T1]{fontenc}
\usepackage{xparse}
\usepackage{indentfirst}
\usepackage{svg}

\usepackage{graphicx}
\graphicspath{ {./img/} {../img/} }


\setlist{topsep=1em, itemsep=0.4em, parsep=0em}
\setlength{\parindent}{16pt}
\setlength{\parskip}{6pt}

\makeatletter
\newcommand{\thetitle}{\@title}
\newcommand{\theauthor}{\@author}
\newcommand{\thedate}{\@date}
\makeatother

\newcommand{\version}{1.0 }
\newcommand{\bigspace}{\vspace{1.9cm}}
\newcommand{\smallspace}{\vspace{0.5cm}}

\NewDocumentCommand{\image}{ m O{1} O{\linewidth} }{
    \begin{center}
        \resizebox{#3}{!}{
            \ifnum#2=0
                \includegraphics{#1}
            \else
                \setlength{\fboxsep}{0pt}%
                \setlength{\fboxrule}{1pt}%
                \fbox{%
                    \includegraphics{#1}%
                }
            \fi
        }
    \end{center}
}

\NewDocumentCommand{\svgimage}{ m O{1} O{\linewidth} }{
    \begin{center}
        \resizebox{#3}{!}{
            \ifnum#2=0
                \includesvg{#1}
            \else
                \setlength{\fboxsep}{0pt}%
                \setlength{\fboxrule}{1pt}%
                \fbox{%
                    \includesvg{#1}%
                }
            \fi
        }
    \end{center}
}

\renewcommand{\labelitemii}{$\circ$}

\usepackage{hyperref}
\hypersetup{
    pdftitle={\thetitle},
    pdfauthor={\theauthor},
    pdfsubject={SRS for Arion Software}
    pdfkeywords={SRS, Software, Specification, Arion, Flashcards, Spaced Repetition},
    colorlinks=true,   % false: boxed links; true: colored links
    linkcolor=purple,  % color of internal links
    citecolor=magenta, % color of links to bibliography
    filecolor=orange,  % color of file links
    urlcolor=cyan,     % color of external links
    linktoc=page       % only page is linked
}


\begin{document}


\part*{Software Requirements Specification}

\begin{flushright}
    \rule{\linewidth}{5pt}
    \vskip 1cm
    \begin{bfseries}
        \Huge
        SOFTWARE REQUIREMENTS\\
        SPECIFICATION\\
        \smallspace
        for\\
        \smallspace
        Arion\\
        \bigspace

        \LARGE
        Version \version \\
        \smallspace
        Prepared by \theauthor\\
        \smallspace
        \thedate\\
    \end{bfseries}
\end{flushright}

\tableofcontents
\newpage


\chapter{Introduction}

\section{Purpose}
This document details the plans for implementation of the spaced repetition software Arion.
It describes the functional and non-functional requirements of Arion, providing documentation for users
and the software author.
It establishes prerequisites for the software to function, and uses a use case diagram to explain the system.

\section{Document Conventions}
This document was created based on a Latex template for Software Requirements Specification documents that
respects the IEEE standard.

\section{Intended Audience and Reading Suggestions}
This document is intended for:
\begin{itemize}
    \item Users, who want to use Arion to efficiently memorize information.
    \item Programmers, who are interested in furthering the software by developing it or
fixing bugs. 
\end{itemize}

\section{Product Scope}
Arion is a program used to memorize information. Users can create, manage, and study a list of flashcards.
Arion allows users to build their memory over time, allowing them to remember concepts over long periods of time.

\section{References}
% flush left because this text is not intended to fill the whole line, so it throws a warning
\begin{flushleft} 
    GitHub page: \\
    \url{https://github.com/Some-Guy-2017/arion} \\
    LaTeX Software Requirements Specification template: \\
    \url{https://github.com/jpeisenbarth/SRS-Tex} \\
    Spaced Repetition History: \\
    \url{https://files.eric.ed.gov/fulltext/EJ1143520.pdf} \\
    Information on the Leitner System: \\
    \url{https://mindedge.com/learning-science/the-leitner-system-how-does-it-work/} \\
\end{flushleft}


\chapter{Overall Description}

\section{Product Perspective}
Arion was built to be small and simple.
Arion has minimal features and a simple GUI, allowing users of varying technical proficiency to use Arion,
and making modification by users with programming experience easy.
Arion is an open source, standalone application made for a senior year high school final project.
It runs on any system that supports Java, such as Windows, macOS, and Linux.

\section{Product Functions}
\label{sec:Product-Functions}
File:
\begin{itemize}
    \item Load: Loads flashcards from the database.
    \item Save: Saves flashcards to the database.
\end{itemize}

Edit:
\begin{itemize}
    \item Browse: Displays the flashcards in a table.
    \begin{itemize}
        \item Delete: Deletes a flashcard of the user's choice.
        \item Edit: Edits a flashcard with new fields specified by the user.
    \end{itemize}
    \item Add: Adds a flashcard to the list with fields specified by the user.
\end{itemize}

View:
\begin{itemize}
    \item Study: The user reviews the due flashcards.
    \item Sort: Sorts the flashcards in a manner specified by the user.
\end{itemize}

Help:
\begin{itemize}
    \item Guide: Displays a user guide explaining how to use Arion.
    \item About: Summarizes Arion's functionality.
\end{itemize}

Quit: Quits Arion.

\section{User Classes and Characteristics}
\begin{itemize}
    \item Language learners, who want to use Arion to study new or old vocabulary.
    \item Students, who want to memorize content for their classes.
    \item Programmers, who are interested in furthering the software by developing it or fixing bugs.
\end{itemize}

\section{Operating Environment}

\begin{flushleft} % text does not fill the line, so must flush left to avoid error.
    Arion is designed to work any PC equipped with a JVM. Valid Operating Systems include:
    \begin{itemize}
        \item Windows Vista
        \item Windows 7
        \item Windows 8
        \item Windows 10
        \item Windows 11
        \item Mac OS
        \item Linux
    \end{itemize}
\end{flushleft}

\section{Design and Implementation Constraints}
Arion is implemented in Java, using Java SWING and AWT for the Graphical User Interface (GUI).
It reads and writes to the user’s machine, and requires a JVM to be installed.

\section{User Documentation}
\begin{flushleft}
    A user guide is included in the Arion help menu. \\
    Arion has a user manual: \\
    \url{https://drive.google.com/file/d/1R3L0F9FErr2sn22dniuIB0acMaw8OTUr/view?usp=sharing}
\end{flushleft}

\section{Assumptions and Dependencies}
Arion is written in Java, so requires a JVM version of at least version 17 to run.


\chapter{External Interface Requirements}

\section{User Interfaces}
\begin{itemize}
    \item
        Main Screen: The default screen that gets displayed when the user starts Arion.
        It includes a Study button that enters Study mode,
        and an Add button that enters Add mode.
    \item
        Menu: A menu bar is displayed at the top of the window for users to perform various actions.
        The menus and their actions are detailed in the \hyperref[sec:Product-Functions]{Product Functions} section.
    \item 
        Browse: Has a table listing the flashcards, with editable fields, an Update button to confirm changes,
        and a Delete button to delete entries. A Back button allows users to return to the main screen.
    \item
        Study: Displays the flashcards and a bar with buttons.
        If the front of a flashcard is shown, the bar contains a button to flip the card. 
        If the back of the card is show, the bar contains a correct and an incorrect button for users to self-score
        their memory. It has a back button to return to the main screen.
    \item 
        Sort: Has a drop-down allowing users to sort based on flashcard fronts or backs,
        and a drop-down for sorting in lexicographical or reverse lexicographical order. 
        Also has a button for confirming the sort, and a back button to return to the main screen.
    \item 
        Guide: A pop-up window with a number of screens, each detailing the usage of a
        different part of Arion. Has an exit button to return to the main program.
    \item 
        About: A pop-up window detailing the Arion program, including its purpose and development.
        Has an exit button to return to the main program.
    \item 
        Add: A screen with two labeled editable text fields for users to enter the front and
        back of a new card. Also has an Add button to add the flashcard to the list, and
        a back button to return.
\end{itemize}

\section{Hardware Interfaces}
Since Arion runs in a JVM, all hardware must be able to run Java.

\section{Software Interfaces}
Arion is written in Java, so requires a JVM version 17 or higher to be installed on the system.

\section{Communications Interfaces}
Arion has no external communications with other services.


\chapter{System Features}

\section{Load}
    \subsection*{Stimulus/Response Sequence}
        \begin{flushleft}
            \makebox[1.7cm][l]{Stimulus:} The user clicks the Load menu item. \\
            \makebox[1.7cm][l]{Stimulus:} The user opens Arion. \\
            \makebox[1.7cm][l]{Response:} Arion loads the flashcards from the database. \\
        \end{flushleft}

    \subsection*{User Requirements}
        \begin{itemize}
            \item If there are preexisting flashcards, the user must confirm overwriting them.
        \end{itemize}

    \subsection*{System Requirements}
        \begin{itemize}
            \item Arion must have permission to read from the database.
            \item Arion must have enough memory for the flashcards.
        \end{itemize}

\section{Save}
    \subsection*{Stimulus/Response Sequence}
        \begin{flushleft}
            \makebox[1.7cm][l]{Stimulus:} The user clicks the Save menu item. \\
            \makebox[1.7cm][l]{Stimulus:} The user closes Arion. \\
            \makebox[1.7cm][l]{Response:} Arion saves the program flashcards to the database.
        \end{flushleft}

    \subsection*{User Requirements}
        \begin{itemize}
            \item The user must confirm overwriting their previously stored flashcards.
        \end{itemize}

    \subsection*{System Requirements}
        \begin{itemize}
            \item Arion must have permission to write to the database.
        \end{itemize}

\section{Browse}
    \subsection*{Stimulus/Response Sequence}
        \begin{flushleft}
            \makebox[1.7cm][l]{Stimulus:} The user clicks the Browse menu item. \\
            \makebox[1.7cm][l]{Response:} Arion displays the Browse screen. \\
        \end{flushleft}

    \subsection*{User Requirements}
    There are no user requirements.

    \subsection*{System Requirements}
        \begin{itemize}
            \item There must be flashcards to display.
        \end{itemize}

\section{Delete}
    \subsection*{Stimulus/Response Sequence}
        \begin{flushleft}
            \makebox[1.7cm][l]{Stimulus:} The user clicks the delete button next to the flashcard entry in the Browse table. \\
            \makebox[1.7cm][l]{Response:} Arion deletes the selected flashcard. \\
        \end{flushleft}

    \subsection*{User Requirements}
        There are no user requirements.

    \subsection*{System Requirements}
        \begin{itemize}
            \item Arion must be in browse mode.
        \end{itemize}

\section{Edit}
    \subsection*{Stimulus/Response Sequence}
        \begin{flushleft}
            \makebox[1.7cm][l]{Stimulus:} The user edits a field in the flashcard table. \\
            \makebox[1.7cm][l]{Response:} Arion stores the edited information. \\
            \smallspace
            \makebox[1.7cm][l]{Stimulus:} The user presses the Update button next to the flashcard in the browse screen table. \\
            \makebox[1.7cm][l]{Response:} Arion writes the edited information to the internally stored list of flashcards. \\
        \end{flushleft}
    \subsection*{User Requirements}
        \begin{itemize}
            \item The user must enter the new information.
            \item The user must click the Update button to update the information.
        \end{itemize}

    \subsection*{System Requirements}
        \begin{itemize}
            \item Arion must be in browse mode.
        \end{itemize}

\section{Add}
    \subsection*{Stimulus/Response Sequence}
        \begin{flushleft}
            \makebox[1.7cm][l]{Stimulus:} The user clicks the Add menu item. \\
            \makebox[1.7cm][l]{Response:} Arion displays the Add screen. \\
            \smallspace
            \makebox[1.7cm][l]{Stimulus:} The user clicks the Add button in the add screen. \\
            \makebox[1.7cm][l]{Response:} Arion adds the new contact to the contact list. \\
        \end{flushleft}

    \subsection*{User Requirements}
        \begin{itemize}
            \item The user must enter the new information for the contact.
            \item The user must click the Add button to confirm adding the new contact.
        \end{itemize}

    \subsection*{System Requirements}
        \begin{itemize}
            \item Arion must have enough memory for the new flashcard.
        \end{itemize}

\section{Study}
    \subsection*{Stimulus/Response Sequence}
        \begin{flushleft}
            \makebox[1.7cm][l]{Stimulus:} The user clicks the Study button  \\
            \makebox[1.7cm][l]{Response:} Arion enters Study mode. \\
        \end{flushleft}

    \subsection*{User Requirements}
        \begin{itemize}
            \item The user must flip the flashcards.
            \item The user must self-score their knowledge of their memory.
        \end{itemize}

    \subsection*{System Requirements}
        \begin{itemize}
            \item There must be flashcards to study.
        \end{itemize}

\section{Sort}
    \subsection*{Stimulus/Response Sequence}
        \begin{flushleft}
            \makebox[1.7cm][l]{Stimulus:} The user clicks the Sort menu item. \\
            \makebox[1.7cm][l]{Response:} Arion displays the Sort screen. \\
            \smallspace
            \makebox[1.7cm][l]{Stimulus:} The user enters the desired sorting configuration. \\
            \makebox[1.7cm][l]{Response:} Arion stores the configuration. \\
            \smallspace
            \makebox[1.7cm][l]{Stimulus:} The user clicks the Sort button in the Sort screen. \\
            \makebox[1.7cm][l]{Response:} Arion sorts the flashcards based on the user's options. \\
        \end{flushleft}
    \subsection*{User Requirements}
        \begin{itemize}
            \item The user must enter whether to sort based on card's front, back, review dates, or review interval.
            \item The user must enter whether to sort the cards from smallest value to largest value, or largest value to smallest value.
            \item The user must confirm their selection by clicking the Sort button.
        \end{itemize} 

    \subsection*{System Requirements}
    \begin{itemize}
        \item There must be flashcards to sort.
    \end{itemize}


\section{Guide}
    \subsection*{Stimulus/Response Sequence}
        \begin{flushleft}
            \makebox[1.7cm][l]{Stimulus:} The user clicks the Guide menu item. \\
            \makebox[1.7cm][l]{Response:} Arion displays the Guide pop-up window. \\
        \end{flushleft}

    \subsection*{User Requirements}
    There are no user requirements.

    \subsection*{System Requirements}
        \begin{itemize}
            \item Arion must have permission to read from the Guide text file.
        \end{itemize}

\section{About}
    \subsection*{Stimulus/Response Sequence}
        \begin{flushleft}
            \makebox[1.7cm][l]{Stimulus:} The user clicks the About menu item. \\
            \makebox[1.7cm][l]{Response:} Arion displays the About pop-up window. \\
        \end{flushleft}

    \subsection*{User Requirements}
    There are no user requirements.

    \subsection*{System Requirements}
    There are no system requirements.

\section{Quit}
    \subsection*{Stimulus/Response Sequence}
        \begin{flushleft}
            \makebox[1.7cm][l]{Stimulus:} The user clicks the Quit menu item. \\
            \makebox[1.7cm][l]{Stimulus:} The user clicks the window exit button. \\
            \makebox[1.7cm][l]{Response:} Arion quits. \\
        \end{flushleft}

    \subsection*{User Requirements}
    There are no user requirements

    \subsection*{System Requirements}
    There are no system requirements.


\chapter{Milestones}
    \begin{enumerate}[1.]
        \item Create a detailed Software Requirements Specification.
        \item Create class and sequence diagrams detailing the project.
        \item Create a minimum viable product.
        \item Add remaining non-functional requirements.
        \item Perform unit and system testing.
        \item Deploy Arion.
    \end{enumerate}


\chapter{Key Resource Requirements}

\section{Major tasks}
Since this is being developed by a single developer, all tasks are assigned to Joseph Hare.
The tasks are:
\begin{itemize}
    \item Requirement documentation.
    \item Class and sequence diagram creation.
    \item GUI implementation.
    \item Back end implementation.
    \item Testing.
    \item User manual creation.
\end{itemize}


\chapter{Other Requirements}

\section{Performance Requirements}
\begin{itemize}
    \item GUIs shall load within 10ms, otherwise Arion will feel unresponsive to users.
    \item Sorting shall be performed within 20ms for 5,000 flashcards.
    \item Adding, editing, and deleting flashcards shall take no more than 5ms.
    \item Displaying flashcards shall take no more than 5ms.
        Since this process occurs frequently, a high delay would irritate users.
\end{itemize}

\section{Software Quality Attributes}
\begin{itemize}
    \item Arion shall be modular and written in an object-oriented programming language,
        to allow for future updates and ease of creation.
    \item Arion shall be programmed in an easily portable programming language,
        such as Java, to allow for future porting to other systems.
    \item Arion shall be reliable and robust, so no unexpected inputs will
        make it fail.
\end{itemize}


\chapter{Appendices}

\section{Glossary}
\begin{itemize}
    \item Flashcard: A card containing a prompt on the front and information on the back.
        They are used to memorize information.
        Learners read the prompt on the front, and attempt to use that context to recall the information on the back.
        For instance, a prompt could be "Bonjour" and the back could be "Hello"; the learner has to remember the translation into English.
    \item Spaced repetition: A way to memorize information. New and difficult information
        is show more frequently than easy and older information, mimicking the pattern
        by which people forget information to optimize learning rate.
    \item Java: A multi-platform object-oriented programming language.
    \item Multi-platform: Able to run on many platforms without needing to alter the code.
    \item Object-oriented: A style of programming where data and code are organized into
        separate "objects."
    \item Graphical User Interface (GUI): A visible digital interface where users can 
        interact with the program using buttons, icons, etc.
    \item Java Abstract Window Toolkit (AWT): Java's first graphics toolkit, allowing developers
        to provide a GUI for their programs.
    \item Java SWING: A newer graphical toolkit for Java that can emulate the look of multiple platforms.
    \item Virtual Machine (VM): A computer implemented in software, as opposed to hardware.
    \item Java Virtual Machine (JVM): Java's VM used to run compiled Java programs.
    %\item Oracle: An American technology company that owns and maintains Java.
    \item Review Interval: The length of time until a flashcard is due.
    \item Use Case Actor: Entities that can perform actions. For example, users can load, edit, and add flashcards, and databases can store and retrieve flashcards.
    \item Functional Requirement: A fundamental feature of a system.

\end{itemize}

\section{Analysis Models}
\subsection{Use Case Diagram}
The following is a use case diagram showing how actors interact with functional requirements:
\image{use-case-diagram}

\subsection{Class Diagram}
The following is a class diagram showcasing the breakdown of classes, their methods, method parameters and return types, and how they interact.
{\fontsize{8}{10}\selectfont
    \textbf{
        \svgimage{class-diagram}
    }
}

\subsection{Sequence Diagrams}
The sequence diagrams describe how objects interact in order to perform a given feature.

\subsubsection*{Load Sequence Diagram}
To begin the Load feature, the user clicks the Load menu item.
This causes ArionDisplay to call the Load menu's menuCallback, which calls the loading functionality in loadFlashcards.
Arion confirms overwriting the flashcards stored in memory with the user, then uses the Database class to parse the flashcards from a data file.
The Database then returns these flashcards, which are written into memory.
\image{load-sd}[0]

\subsubsection*{Save Sequence Diagram}
The method calls for saving flashcards is similar to those of the Load functionality;
    its menu item is clicked, ArionDisplay runs the corresponding callback, Arion calls a function to perform the desired functionality,
    and Arion confirms the User's choice.
In this case, Arion is confirming the overwriting of the flashcards stored in the data file, rather than those currently stored in memory.
Arion then uses the Database class to format and write the flashcards to the data file.
\image{save-sd}[0]

\subsubsection*{Browse Sequence Diagram}
After the Browse menu item is clicked and the corresponding menuCallback is run,
    Arion calls the displayBrowseScreen method of ArionDisplay to display the browse screen.
It passes the flashcards, since these are displayed in the screen, and an edit and delete callback.
When ArionDisplay creates the delete and edit buttons, it sets their ActionListeners to perform one of these callbacks,
    which perform functionality with data in the Arion class.
\image{browse-sd}[0]

\subsubsection*{Delete Sequence Diagram}
When the delete button is clicked next to a flashcard in the Browse screen,
    ArionDisplay calls the deleteCallback passed in the displayBrowseScreen method, passing the index of the flashcard to delete.
This then calls the deleteFlashcard method of Arion with the flashcard index,
    which deletes the selected card while preserving order.
\image{delete-sd}[0]

\subsubsection*{Edit Sequence Diagram}
In the Edit feature, when the Update button is clicked next to a flashcard in the Browse screen,
    ArionDisplay runs the editCallback passed in the displayBrowseScreen method.
This callback takes the index of the flashcard to edit, as well as the new fields to replace the previous fields with.
The Arion class replaces the flashcard's fields in the editFlashcard method, then returns.
\image{edit-sd}[0]

\subsubsection*{Add Sequence Diagram}
There are two ways the Add feature can start:
\begin{itemize}
    \item If the user clicks the main screen Add button, ArionDisplay calls displayAddScreen with
        the addCallback passed in displayMainScreen.
    \item If the Add menu is clicked, ArionDisplay calls the menuCallback, which then calls displayAddScreen with
        the callback it passed in displayMainScreen.
\end{itemize}
After either of these has occurred, the user enters the information for the new flashcard in the Add screen, then clicks the Add screen Add button.
This triggers ArionDisplay to call the addCallback, which calls addFlashcard with the new flashcard fields.
The flashcard is constructed from the fields and added to the list, then the function return.
\image{add-sd}[0]

\subsubsection*{Study Sequence Diagram}
\noindent
The Study feature can be started in two ways:
\begin{itemize}
    \item The user clicks the Study button in the main screen, which calls the studyCallback passed in displayMainScreen.
    \item The user clicks the Study Menu, which calls its corresponding menuCallback.
\end{itemize}
Since both of these callbacks perform the same functionality,
    they are both represented as studyCallback.run() in the sequence diagram.
    
The callbacks run studyFlashcards, which prepares a list of due flashcards, and writes them to Arion's dueFlashcards field.
It then calls displayStudyScreen with the first of these flashcards, with
    front set to true so ArionDisplay will show the front of the card, 
    and a reviewCallback for ArionDisplay to execute once the card has been reviewed.
Now, all functions have completed, so they return.

Once the Flip button on the front of the Study screen is clicked,
it activates an ActionListener that calls displayStudyScreen with the same parameters, but with front now set to false.
When either of the Correct or Incorrect Review buttons have been clicked on the back of the flashcard,
    an ActionListener calls the reviewCallback with the success value (true for the Correct button, false for the Incorrect button).
    
The reviewCallback calls the updateReviewDate method on the first flashcard in the dueFlashcards list with the success value it was passed.
Then, it either removes the flashcard if it was recalled successfully, or moves the flashcard to the end of the list if it was not.
Lastly, it checks if there are still due flashcards.
If not, it returns; otherwise, it calls displayStudyScreen with the next flashcard, and the cycle repeats until all flashcards have been studied.
\image{study-sd}[0]

\subsubsection*{Sort Sequence Diagram}
The Sort feature starts when the user clicks the Sort menu item.
This causes ArionDisplay to call the corresponding menuCallback,
    which calls the displaySortScreen method with a callback for when the user clicks the Sort button.
After the user clicks the Sort button and the callback is run,
    Arion calls the sortFlashcards method with the flashcard field to sort by and whether to sort it in reversed order.
The flashcards are sorted according to the user's configuration, and the methods return.
\image{sort-sd}[0]

\subsubsection*{Guide Sequence Diagram}
The Guide feature is started once the user clicks the Guide menu item.
This causes ArionDisplay to call displayGuidePage with a value 0, which displays the first guide page in a pop-up window.
The page content is then gotten from the GuideReader by calling readPage with the page index, which starts at 0.
It then displays the guide content, and returns.

When the user clicks the Next button in the Guide window, the ActionListener calls displayGuidePage with the next page index.
The same process occurs, where ArionDisplay gets the GuideElements from the GuideReader and displays them.

Lastly, if the user clicks the Previous button, the ActionListener calls the displayGuidePage with the previous page index,
and repeats the same process to display it.
\image{guide-sd}[0]

\subsubsection*{About Sequence Diagram}
About requires minimal method calls: when its menu is clicked, the callback is run, calling displayAboutScreen.
This generates a pop-up window with the About screen description, then returns.
\image{about-sd}[0]

\subsubsection*{Quit Sequence Diagram}
Quit is similarly simple to About. It can be activated in two ways:
\begin{itemize}
    \item If the Quit menu is clicked, its callback is run, which calls the quit method.
        The quit method saves the flashcards, and then quits.
    \item If the window exit button is clicked, Arion directly runs the quit method, which performs the same process as when the Quit menu is clicked.
\end{itemize}
\image{quit-sd}[0]

\section{Project Proposal}

\subsection{Problem Statement}
Traditional study methods often result in forgetting information over long periods of time.
Memorization over years, or even a lifetime, is crucial for
speaking a second language, entering a new profession, and many more scenarios.
A standardized approach to learning information is thus needed.

\subsection{Objectives}
Arion primarily aims to provide a spaced-repetition system to allow learners to 
systemically built their knowledge over long periods of time. Arion's other objectives are to:
\begin{itemize}
    \item Provide a friendly user interface for managing flashcards.
    \item Be open-source to allow free usage.
    \item Work quickly and robustly for positive user experience.
    \item Allow future cross-platform availability.
    \item Be written in a modular way for future expansions.
    \item Be written clearly using DRY principals for easy development and updates.
\end{itemize}

\subsection{Methodology}
The development of Arion will follow the following process:
\begin{enumerate}[1.]
    \item Document the specification for the development process using a Software Requirements Specification and use cases.
    \item Detail the specifics of Arion's implementation using class and sequence diagrams.
    \item Implement the back-end of Arion (card creation, deletion, editing, etc.).
    \item Implement the front-end of Arion (user interface, buttons, labels, etc.).
    \item Rigorously test the product using unit and system tests.
    \item Create a user manual.
    \item Deploy Arion.
\end{enumerate}

\subsection{History of Product}
Spaced repetition started in 1885, when Hermann Ebbinghaus hypothesized that the rate at which people forget information increases exponentially
over time. Further study realized the following two principals:
\begin{enumerate}[1.]
    \item Recalling information leads to better memory than being shown the item.
    \item Recalling information after a delay increases memory retention compared
        to recall soon after learning the item.
\end{enumerate}
The optimal strategy is to have learners review the information just before it will
be forgotten. Several estimations of when people will forget knowledge have been made;
Arion is based on the Leitner system of temporally space flashcards, with the modification that
correctly reviewed cards appear again after $\lfloor 1.6^x \rfloor$ days
(where $x$ is the number of correct reviews).



\part*{Use Cases}

\addtocontents{toc}{\protect\setcounter{tocdepth}{-1}} % hide chapters from ToC
\setcounter{chapter}{0} % reset numbering

\chapter{Arion - Load}

\section{Description}
Load parses the database (always at the same place relative to the program executable) for flashcards,
then loads them into memory.

\section{Actors}
\begin{itemize}
    \item User
    \item Database
\end{itemize}

\section{Preconditions}
\begin{itemize}
    \item There is a database for Arion to read from.
    \item Arion has permission to read from the database.
\end{itemize}

\section{Expected Flow of Events}
\begin{enumerate}[1.]
    \item User clicks the Load menu item.
    \item Ask the user to confirm overwriting the previous flashcards.
    \item Open the database for reading.
    \item Parse the database for flashcards.
    \item Validate each flashcard.
    \item Write each flashcard into a local array.
    \item Close the database.
\end{enumerate}

\section{Alternative Flow of Events}

    \subsection{No Database}
    If in step 3 the database does not yet exist:
    \begin{enumerate}
        \item Display an error message to the user.
        \item End the use case with a failure condition.
    \end{enumerate}

    \subsection{Invalid Flashcard}
    If in step 5 a flashcard is formatted incorrectly:
    \begin{enumerate}
        \item Display an error message stating that the database is formatted incorrectly.
        \item End the use case with a failure condition.
    \end{enumerate}

\section{Key Scenarios}
    \subsection{Loading Flashcards}
    The flashcards are being loaded and parsed from the database.
    This must be performed quickly, as during this time the user has no available actions.
    If they were to wait for a long period of time without being able to use the program,
    it would impair user experience.

    \subsection{Permission Error}
    Arion does not have permission to read from the database.
    In this case, it must be made clear to the user that the flashcards could not
    be loaded, and some permission altering must be performed on their end in order
    for this crucial functionality to be used.
    
\section{Post-Conditions}
    \subsection{Successful Condition}
    The flashcards were loaded into memory.

    \subsection{Failure Condition}
    The flashcards were not loaded, and a failure message was displayed.


\chapter{Arion - Save}

\section{Description}
Save saves the flashcards loaded in memory to the database.

\section{Actors}
\begin{itemize}
    \item User
    \item Database
\end{itemize}

\section{Preconditions}
\begin{itemize}
    \item Arion must have permission to write to the database.
    \item The user must confirm the overwriting of any preexisting flashcards.
\end{itemize}

\section{Expected Flow of Events}
\begin{enumerate}[1.]
    \item User clicks the Save menu item.
    \item Ask the user to confirm overwriting the flashcards stored in the database.
    \item Open the database for writing.
    \item Write the header into the database.
    \item Write each flashcard to the database.
    \item Close the database.
\end{enumerate}

\section{Alternative Flow of Events}

    \subsection{User Declines}
    If in step 1 the user declines to overwrite the flashcards:
    \begin{enumerate}
        \item Display an error message to the user.
        \item End the use case with a failure condition.
    \end{enumerate}

    \subsection{Database Failure}
    If in step 2 the database cannot be opened for writing:
    \begin{enumerate}[1.]
        \item Display an error message.
        \item End the use case with a failure condition.
    \end{enumerate}

\section{Key Scenarios}
    \subsection{Permission Error}
    Arion does not have permission to write to the database.
    In this case, it must be made clear to the user that the flashcards could not
    be saved, and some permission altering must be performed on their end in order
    for this crucial functionality to be used.

\section{Post-Conditions}
    \subsection{Successful Condition}
    The flashcards were saved to the database.
    
    \subsection{Failure Condition}
    The database is unchanged, and an error message was displayed.

\chapter{Arion - Browse}

\section{Description}
Browse displays the flashcards in a table with editable fields. This lets users
view, edit, and delete their flashcards.

\section{Actors}
\begin{itemize}
    \item User
\end{itemize}

\section{Preconditions}
\begin{itemize}
    \item There must be flashcards to display.
\end{itemize}

\section{Expected Flow of Events}
\begin{enumerate}[1.]
    \item User clicks the Browse menu item.
    \item Check that there are flashcards to display.
    \item Construct a GUI table.
    \item Fill the table with the flashcard fields.
    \item Display the table.
\end{enumerate}

\section{Alternative Flow of Events}

    \subsection{No Flashcards}
    If in step 2 there are no flashcards:
    \begin{enumerate}
        \item Display to the user that there are no flashcards.
        \item End the use case with a failure condition.
    \end{enumerate}

\section{Key Scenarios}
    \subsection{Parsing and Rendering of Flashcards}
    Flashcard parsing and rendering must be performed quickly, as the user
    must wait for this to complete before proceeding to their next action.
    A long delay would impair user experience.

\section{Post-Conditions}
    \subsection{Successful Condition}
    The flashcards were displayed.
    
    \subsection{Failure Condition}
    The program displayed that there are no flashcards to display.


\chapter{Arion - Delete}

\section{Description}
Delete removes a flashcard from the list stored in memory.

\section{Actors}
\begin{itemize}
    \item User
\end{itemize}

\section{Preconditions}
\begin{itemize}
    \item Arion is in browse mode.
    \item There must be at least one flashcard.
\end{itemize}

\section{Expected Flow of Events}
\begin{enumerate}[1.]
    \item User clicks the Delete button next to a flashcard.
    \item Ask the user to confirm their selection.
    \item Delete the selected flashcard from the list.
    \item Update the displayed table.
\end{enumerate}

\section{Alternative Flow of Events}

    \subsection{User Declines}
    If in step 1 the user declines the selection,
    end the use case with a failure condition.

\section{Key Scenarios}
    \subsection{Confirmation of Card Deletion}
    The confirmation of deleting the card must work properly,
    as Arion is likely the only location of the information stored on the card.
    Accidentally deleting the card would either permanently lose this information,
    or force the user to reload from the database, removing their previous work.
    Furthermore, failing to delete unwanted cards would disable this feature entirely.
    Thus, for optimal user experience, this part of the feature must be implemented flawlessly.

\section{Post-Conditions}
    \subsection{Successful Condition}
    The flashcard was deleted.
    
    \subsection{Failure Condition}
    The flashcard was not deleted.


\chapter{Arion - Edit}

\section{Description}
Edit modifies a flashcard's fields to the user's choice.
The only fields users cannot edit are review date and review interval.

\section{Actors}
\begin{itemize}
    \item User
\end{itemize}

\section{Preconditions}
\begin{itemize}
    \item Arion must be in Browse mode.
    \item There must be at least one flashcard.
\end{itemize}

\section{Expected Flow of Events}
\begin{enumerate}[1.]
    \item User enters the new fields for the flashcard.
    \item User clicks the Update button.
    \item Update the flashcard's fields with the new values.
\end{enumerate}

\section{Alternative Flow of Events}
    There is no alternative flow of events.

\section{Key Scenarios}
    \subsection{User Enters New Information}
    When the user is entering information, two things are crucial:
    the process feels intuitive, and it is clear the user has to update the flashcard.
    If the process is unintuitive and difficult to use,
    editing many flashcards can become a chore,
    and ultimately dissuade users from using the program.
    Additionally, if users do not understand they have to press the Update button to
    confirm the modification to the flashcard's fields, they could lose their edits.
    This would make Arion frustrating to use and impair user experience.
    Thus, some sort of marker (such as an asterisk) should be used to indicated
    unsaved edits.

\section{Post-Conditions}
    \subsection{Successful Condition}
    The flashcard's fields were replaced with the new values.
    
    \subsection{Failure Condition}
    The flashcard's fields are unchanged.

\chapter{Arion - Add}

\section{Description}
Add displays a screen where the user can input the back and front of a new flashcard,
then press an Add button to add it to the list of flashcards.
The flashcard's date is set to the current day.

\section{Actors}
\begin{itemize}
    \item User
\end{itemize}

\section{Preconditions}
    There are no preconditions.

\section{Expected Flow of Events}
\begin{enumerate}[1.]
    \item User clicks the Add menu item or Add button.
    \item Display the Add screen.
    \item User enters the flashcard fields.
    \item User clicks the Add button.
    \item The flashcard date is set to the current date.
    \item Append the flashcard to the flashcard list.
    \item Return to the main screen.
\end{enumerate}

\section{Alternative Flow of Events}

    \subsection{Back Button}
    If while the Add screen is displayed the user clicks the back button,
    return to the main screen.

\section{Key Scenarios}
    \subsection{User Enters New Information}
    When the user is entering information, it is crucial that the process feels intuitive.
    Adding flashcards is a crucial feature of Arion; without this feature, the program would be useless.
    Furthermore, hundreds of flashcards may be added; thus, adding must feel easy and intuitive.

\section{Post-Conditions}
    \subsection{Successful Condition}
    The flashcard was added to the flashcard list.
    
    \subsection{Failure Condition}
    The flashcard was not added to the flashcard list.

\chapter{Arion - Study}

\section{Description}

Study enters study mode. First, the user is presented with the front of a flashcard.
After they have attempted a recall of the information on the back, they flip the card,
then self-score their memory. If they succeed, the amount of time until the user reviews the card
again becomes $1.6$ times longer (e.g. if they reviewed in on January 1st, then correctly reviewed
it 5 days later on January 6th, then they would again review it on January 14th, $5\cdot 1.6=8$ days later).
If they fail, the flashcard is due the next day. This is repeated for all due flashcards.

\section{Actors}
\begin{itemize}
    \item User
\end{itemize}

\section{Preconditions}
\begin{itemize}
    \item There must be due flashcards today.
\end{itemize}

\section{Expected Flow of Events}
\begin{enumerate}[1.]
    \item User clicks the study button in the main screen.
    \item Prepare a list of the flashcards due today.
    \item Check that the list is not empty.
    \item Display the Study screen.
    \item For each flashcard in the list of cards:
    \begin{enumerate}[a.]
        \item Show the front of the card.
        \item User flips the card
        \item User self-scores their performance.
        \item If the user recalls correctly:
        \begin{enumerate}[i)]
            \item Multiply the review interval by $1.6$.
            \item Update the review date to be the current date plus the review interval.
            \item Remove the card from the due list.
        \end{enumerate}
        \item Otherwise, if the user has recalled incorrectly:
        \begin{enumerate}[i)]
            \item Set the review interval to $1$.
            \item Set the review date to today.
            \item Move the card to the end of the due list.
        \end{enumerate}
    \end{enumerate}
    \item Return to the main screen.
    \item Write a congratulatory message on the screen. 
\end{enumerate}

\section{Alternative Flow of Events}
    \subsection{User Exits}
    If the user exits early from studying, before all flashcards have been reviewed:
    \begin{enumerate}[1.]
        \item Return to the main screen.
        \item End the use case with a failure condition.
    \end{enumerate}
    
    \subsection{No Due Flashcards}
    If in step 3 there are no flashcards due today:
    \begin{enumerate}[1.]
        \item Display to the user that there are no due flashcards to study.
        \item End the use case with a failure condition.
    \end{enumerate}

\section{Key Scenarios}
    \subsection{Due Card Finding}
    The due cards are found before studying.
    If this process is performed poorly, cards due in the future will be reviewed early,
    which defeats the beneficial effects of spaced repetition.
    For optimal user memory, only cards due before or on the current date can be reviewed.

\section{Post-Conditions}
    \subsection{Successful Condition}
    All flashcards have been reviewed.
    
    \subsection{Failure Condition}
    No flashcards have been studied, or some flashcards have not been studied.

\chapter{Arion - Sort}

\section{Description}
Sort displays a screen where users can select to sort the flashcards based on their
front, back, review date, or review interval, and whether to do so from smallest to largest or largest to smallest.
It also has a Sort button for the user to confirm sorting with the
defined configuration.
This feature is helpful for users to parse their flashcards in the Browse screen;
since the order in which due cards are studied is not meaningful to spaced repetition,
it has no significant effect on flashcard review.

\section{Actors}
\begin{itemize}
    \item User
\end{itemize}

\section{Preconditions}
\begin{itemize}
    \item There must be at least one flashcard to sort.
\end{itemize}

\section{Expected Flow of Events}
\begin{enumerate}[1.]
    \item User clicks the Sort menu item.
    \item Check that there are flashcards to sort.
    \item Display the Sort screen.
    \item User enters their sorting configuration.
    \item User clicks the Sort button.
    \item Sort the flashcards.
    \item Return to the main screen.
\end{enumerate}

\section{Alternative Flow of Events}

    \subsection{No Flashcards}
    If in step 2 there are no flashcards:
    \begin{enumerate}
        \item Display to the user that there are no flashcards to sort.
        \item End with a failure condition.
    \end{enumerate}
    
    \subsection{User Exits}
    If the user presses the back button to exit prematurely:
    \begin{enumerate}[1.]
        \item Return to the main screen.
        \item End with a failure condition.
    \end{enumerate}

\section{Key Scenarios}
    \subsection{User Starts the Sorting Process}
    When the user starts the sorting process, it should be completed quickly,
    as the program is frozen during this time.
    It is crucial that users do not have to wait long periods of time for this
    process to complete to promote a positive user experience.

\section{Post-Conditions}
    \subsection{Successful Condition}
    The flashcards were sorted according to the user's configuration.
    
    \subsection{Failure Condition}
    The flashcards were not sorted.


\chapter{Arion - Guide}

\section{Description}
Guide explains to users how to use Arion.
It displays a pop-up window, with each screen explaining detailing a different feature.
It explains how to use the feature, where it is located in the program (i.e. in which menu and which buttons to press), and includes images of the GUIs and GUI elements.
For a list of features, see the \hyperref[sec:Product-Functions]{Product Functions} section.

\section{Actors}
\begin{itemize}
    \item User
\end{itemize}

\section{Preconditions}
There are no preconditions.

\section{Expected Flow of Events}
\begin{enumerate}[1.]
    \item User clicks the Guide menu item.
    \item Read the Guide text from a file.
    \item Create a window containing the guide information, starting on the first screen.
\end{enumerate}

\section{Alternative Flow of Events}

    \subsection{Text File Not Found}
    If in step 2 the text file cannot be found:
    \begin{enumerate}
        \item Display to the user that the file with the Guide text could not be found.
        \item End the use case with a failure condition.
    \end{enumerate}

    \subsection{File Permission Error}
    If in step 2 Arion does not have permission to read from the Guide text file:
    \begin{enumerate}
        \item Display to the user that the file with the Guide text could not be read.
        \item End the use case with a failure condition.
    \end{enumerate}

\section{Key Scenarios}
    \subsection{User Loads Guide}
    The user clicks the Guide menu item.
    It is important for guide navigation to feel comfortable and intuitive,
    as users who require this information are more likely to lack technical proficiency.
    Thus, the guide must be easily understandable by users regardless of technical ability.

\section{Post-Conditions}
    \subsection{Successful Condition}
    The guide window was successfully created.
    
    \subsection{Failure Condition}
    The window could not be created, or the Guide text was not found.
    An error message has been displayed to the user.


\chapter{Arion - About}

\section{Description}
About displays a pop-up window summarizing Arion's functionality.
It outlines its
    purpose, 
    design philosophy,
    development,
    and programming language.

\section{Actors}
\begin{itemize}
    \item User
\end{itemize}

\section{Preconditions}
There are no preconditions.

\section{Expected Flow of Events}
\begin{enumerate}[1.]
    \item User clicks on the About menu item.
    \item Display a pop-up window with the About information.
\end{enumerate}

\section{Alternative Flow of Events}
There is no alternative flow of events.

\section{Key Scenarios}
    \subsection{About Screen}
    The user is viewing the About screen with the About information.

\section{Post-Conditions}
    \subsection{Successful Condition}
    A pop-up window was successfully created.
    
    \subsection{Failure Condition}
    A pop-up window was not created, and an error message has been displayed to the user.


\chapter{Arion - Quit}

\section{Description}
Quit exits the program, saving the flashcards stored in memory to the database.

\section{Actors}
\begin{itemize}
    \item User
    \item Database
\end{itemize}

\section{Preconditions}
There are no preconditions.

\section{Expected Flow of Events}
    \begin{enumerate}[1.]
        \item User clicks the Quit menu item or window exit button.
        \item Perform the Save feature without asking user permission to overwrite the preexisting flashcards.
        \item Close the screen.
        \item Close.
    \end{enumerate}

\section{Alternative Flow of Events}

    \subsection{Database Failure}
    If in step 1 Arion does not have permission to write to the database, end with a failure condition.

\section{Key Scenarios}
    \subsection{Successful Database Write}
    Arion successfully writes to the database.
    This is the most likely scenario, and is the longest part of the closing process;
    thus, it must be performed quickly to ensure the user waits a minimal length of time
    before the program closes.


\section{Post-Conditions}
    \subsection{Successful Condition}
    Arion wrote to the database and closed.
    
    \subsection{Failure Condition}
    Arion could not write to the database, but has still closed.


\end{document}
