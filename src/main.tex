% Based on the template from Jean-Philippe Eisenbarth
% https://github.com/jpeisenbarth/SRS-Tex

\documentclass{scrreprt}
\title{Arion Software Requirements Specification}
\author{Joseph Hare}
\date{\today}

\setlength\parindent{0pt}

\makeatletter
\newcommand{\thetitle}{\@title}
\newcommand{\theauthor}{\@author}
\newcommand{\thedate}{\@date}
\makeatother

\newcommand{\version}{1.0 }
\newcommand{\bigspace}{\vspace{1.9cm}}
\newcommand{\smallspace}{\vspace{0.5cm}}

\usepackage{xcolor}
\usepackage{listings}
\usepackage{underscore}
\usepackage[utf8]{inputenc}
\usepackage[english]{babel}
\usepackage{hyperref}

\usepackage{hyperref}
\hypersetup{
    pdftitle={\thetitle},
    pdfauthor={\theauthor},
    pdfsubject={SRS for Arion Software}
    pdfkeywords={SRS, Software, Specification, Arion, Flashcards, Spaced Repetition},
    colorlinks=true,   % false: boxed links; true: colored links
    linkcolor=black,   % color of internal links
    citecolor=magenta,   % color of links to bibliography
    filecolor=orange, % color of file links
    urlcolor=cyan,     % color of external links
    linktoc=page       % only page is linked
}

\begin{document}

\begin{flushright}
    \rule{\linewidth}{5pt}
    \vskip 1cm
    \begin{bfseries}
        \Huge
        SOFTWARE REQUIREMENTS\\
        SPECIFICATION\\
        \smallspace
        for\\
        \smallspace
        Arion\\
        \bigspace

        \LARGE
        Version \version approved\\
        \smallspace
        Prepared by \theauthor\\
        \smallspace
        \thedate\\
    \end{bfseries}
\end{flushright}

\tableofcontents


\chapter{Introduction}

\section{Purpose}
This document details the planned implementation of the flashcard spaced repetition software Arion.
It describes the functional and non-functional requirements of the system, providing documentation for users
and the software author.
It establishes prerequisites for the software to function, and outlines the planned life cycle of the product.

\section{Document Conventions}
This document was created based on a Latex template for Software Requirement Specification documents that
respect the IEEE standard.

\section{Intended Audience and Reading Suggestions}
\begin{itemize}
    \item Language learners, who want to use Arion to study new vocabulary or remember previously seen words.
    \item Students, who want to memorize content for their classes.
    \item Programmers who are interested in furthering the software by developing it or
fixing bugs. 
\end{itemize}

\section{Project Scope}
Arion is a system used to memorize information. Users can create, manage, and study a list of flashcards.
Arion allows users to build their memory over time, allowing them to remember concepts over long periods of time.

\section{References}
% flush left because this text is not intended to fill the whole line, so 
% it throws a warning (ANNOYING!!)
\begin{flushleft} 
    GitHub page: \\
    \url{https://github.com/Some-Guy-2017/arion} \\
    LaTeX Software Requirements Specification template: \\
    \url{https://github.com/jpeisenbarth/SRS-Tex} \\
\end{flushleft}


\chapter{Overall Description}

\section{Product Perspective}
Arion was built to be small and simple. \href{https://apps.ankiweb.net/}{Anki} is one of the most popular desktop
spaced repetition systems, but infamously has an unintuitive design.
Arion removes unnecessary features and has a simple GUI, improving user experience and making modification by users
with programming experience easy.
Arion is an open source, standalone application made for a senior year high school final project.
It runs on Windows, macOS, and Linux.

\section{Product Functions}
File:
\begin{itemize}
    \item Load: Loads flashcards from the local database.
    \item Save: Saves flashcards to the local database.
\end{itemize}

Edit:
\begin{itemize}
    \item Browse: Displays the flashcards in a table.
    \begin{itemize}
        \item View: View the flashcards.
        \item Delete: Delete a flashcard.
        \item Edit: Edit a flashcard.
    \end{itemize}
\end{itemize}

View:
\begin{itemize}
    \item Study: Study the flashcards.
    \item Sort: Sort the flashcards.
\end{itemize}

Help:
\begin{itemize}
    \item Guide: Display a user guide explaining how to use the program.
    \item About: Summarizes the program's functionality.
\end{itemize}

Quit: Quits the program.

\section{User Classes and Characteristics}
\begin{itemize}
    \item Language learners, who want to use Arion to study new vocabulary or remember previously seen words.
    \item Students, who want to memorize content for their classes.
    \item Programmers who are interested in furthering the software by developing it or fixing bugs.
\end{itemize}

\section{Operating Environment}

\begin{flushleft} % text does not fill the line, so must flush left to avoid error.
    Arion was designed to work on PCs with a Java Virtual Machine. \\

    It works on the following operating systems: \\
    \begin{itemize}
        \item Windows 2000
        \item Windows XP
        \item Windows Vista
        \item Windows 7
        \item Windows 8
        \item Windows 10
        \item Windows 11
        \item Mac OS
        \item Linux
    \end{itemize}
\end{flushleft}

\section{Design and Implementation Constraints}
Arion is implemented in Java, using Java SWING and Abstract Took Kit (AWT) for the Graphical User Interface (GUI).
It reads and writes to the user’s machine, and requires the Java Virtual Machine (JVM) to be installed.

\section{User Documentation}
\begin{flushleft}
    A user guide is included in the Arion help menu. \\
    Arion has a user manual: \\
    \url{https://drive.google.com/file/d/1R3L0F9FErr2sn22dniuIB0acMaw8OTUr/view?usp=sharing}
\end{flushleft}

\section{Assumptions and Dependencies}

$<$List any assumed factors (as opposed to known facts) that could affect the 
requirements stated in the SRS. These could include third-party or commercial 
components that you plan to use, issues around the development or operating 
environment, or constraints. The project could be affected if these assumptions 
are incorrect, are not shared, or change. Also identify any dependencies the 
project has on external factors, such as software components that you intend to 
reuse from another project, unless they are already documented elsewhere (for 
example, in the vision and scope document or the project plan).$>$


\chapter{External Interface Requirements}

\section{User Interfaces}
$<$Describe the logical characteristics of each interface between the software 
product and the users. This may include sample screen images, any GUI standards 
or product family style guides that are to be followed, screen layout 
constraints, standard buttons and functions (e.g., help) that will appear on 
every screen, keyboard shortcuts, error message display standards, and so on.  
Define the software components for which a user interface is needed. Details of 
the user interface design should be documented in a separate user interface 
specification.$>$

\section{Hardware Interfaces}
$<$Describe the logical and physical characteristics of each interface between 
the software product and the hardware components of the system. This may include 
the supported device types, the nature of the data and control interactions 
between the software and the hardware, and communication protocols to be 
used.$>$

\section{Software Interfaces}
$<$Describe the connections between this product and other specific software 
components (name and version), including databases, operating systems, tools, 
libraries, and integrated commercial components. Identify the data items or 
messages coming into the system and going out and describe the purpose of each.  
Describe the services needed and the nature of communications. Refer to 
documents that describe detailed application programming interface protocols.  
Identify data that will be shared across software components. If the data 
sharing mechanism must be implemented in a specific way (for example, use of a 
global data area in a multitasking operating system), specify this as an 
implementation constraint.$>$

\section{Communications Interfaces}
$<$Describe the requirements associated with any communications functions 
required by this product, including e-mail, web browser, network server 
communications protocols, electronic forms, and so on. Define any pertinent 
message formatting. Identify any communication standards that will be used, such 
as FTP or HTTP. Specify any communication security or encryption issues, data 
transfer rates, and synchronization mechanisms.$>$


\chapter{System Features}
$<$This template illustrates organizing the functional requirements for the 
product by system features, the major services provided by the product. You may 
prefer to organize this section by use case, mode of operation, user class, 
object class, functional hierarchy, or combinations of these, whatever makes the 
most logical sense for your product.$>$

\section{System Feature 1}
$<$Don’t really say “System Feature 1.” State the feature name in just a few 
words.$>$

\subsection{Description and Priority}
$<$Provide a short description of the feature and indicate whether it is of 
High, Medium, or Low priority. You could also include specific priority 
component ratings, such as benefit, penalty, cost, and risk (each rated on a 
relative scale from a low of 1 to a high of 9).$>$

\subsection{Stimulus/Response Sequences}
$<$List the sequences of user actions and system responses that stimulate the 
behavior defined for this feature. These will correspond to the dialog elements 
associated with use cases.$>$

\subsection{Functional Requirements}
$<$Itemize the detailed functional requirements associated with this feature.  
These are the software capabilities that must be present in order for the user 
to carry out the services provided by the feature, or to execute the use case.  
Include how the product should respond to anticipated error conditions or 
invalid inputs. Requirements should be concise, complete, unambiguous, 
verifiable, and necessary. Use “TBD” as a placeholder to indicate when necessary 
information is not yet available.$>$

$<$Each requirement should be uniquely identified with a sequence number or a 
meaningful tag of some kind.$>$

REQ-1:	REQ-2:

\section{System Feature 2 (and so on)}


\chapter{Other Nonfunctional Requirements}

\section{Performance Requirements}
$<$If there are performance requirements for the product under various 
circumstances, state them here and explain their rationale, to help the 
developers understand the intent and make suitable design choices. Specify the 
timing relationships for real time systems. Make such requirements as specific 
as possible. You may need to state performance requirements for individual 
functional requirements or features.$>$

\section{Safety Requirements}
$<$Specify those requirements that are concerned with possible loss, damage, or 
harm that could result from the use of the product. Define any safeguards or 
actions that must be taken, as well as actions that must be prevented. Refer to 
any external policies or regulations that state safety issues that affect the 
product’s design or use. Define any safety certifications that must be 
satisfied.$>$

\section{Security Requirements}
$<$Specify any requirements regarding security or privacy issues surrounding use 
of the product or protection of the data used or created by the product. Define 
any user identity authentication requirements. Refer to any external policies or 
regulations containing security issues that affect the product. Define any 
security or privacy certifications that must be satisfied.$>$

\section{Software Quality Attributes}
$<$Specify any additional quality characteristics for the product that will be 
important to either the customers or the developers. Some to consider are: 
adaptability, availability, correctness, flexibility, interoperability, 
maintainability, portability, reliability, reusability, robustness, testability, 
and usability. Write these to be specific, quantitative, and verifiable when 
possible. At the least, clarify the relative preferences for various attributes, 
such as ease of use over ease of learning.$>$

\section{Business Rules}
$<$List any operating principles about the product, such as which individuals or 
roles can perform which functions under specific circumstances. These are not 
functional requirements in themselves, but they may imply certain functional 
requirements to enforce the rules.$>$


\chapter{Other Requirements}
$<$Define any other requirements not covered elsewhere in the SRS. This might 
include database requirements, internationalization requirements, legal 
requirements, reuse objectives for the project, and so on. Add any new sections 
that are pertinent to the project.$>$

\section{Appendix A: Glossary}
%see https://en.wikibooks.org/wiki/LaTeX/Glossary
$<$Define all the terms necessary to properly interpret the SRS, including 
acronyms and abbreviations. You may wish to build a separate glossary that spans 
multiple projects or the entire organization, and just include terms specific to 
a single project in each SRS.$>$

\section{Appendix B: Analysis Models}
$<$Optionally, include any pertinent analysis models, such as data flow 
diagrams, class diagrams, state-transition diagrams, or entity-relationship 
diagrams.$>$

\section{Appendix C: To Be Determined List}
$<$Collect a numbered list of the TBD (to be determined) references that remain 
in the SRS so they can be tracked to closure.$>$

\end{document}
