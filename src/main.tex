% Based on the template from Jean-Philippe Eisenbarth
% https://github.com/jpeisenbarth/SRS-Tex

\documentclass{scrreprt}
\title{Arion Software Requirements Specification}
\author{Joseph Hare}
\date{\today}

\setlength\parindent{0pt}

\makeatletter
\newcommand{\thetitle}{\@title}
\newcommand{\theauthor}{\@author}
\newcommand{\thedate}{\@date}
\makeatother

\newcommand{\version}{1.0 }
\newcommand{\bigspace}{\vspace{1.9cm}}
\newcommand{\smallspace}{\vspace{0.5cm}}

\usepackage{xcolor}
\usepackage{hyperref}
\usepackage[shortlabels]{enumitem}

%\usepackage{listings}
%\usepackage{underscore}
%\usepackage[utf8]{inputenc}
%\usepackage[english]{babel}

\usepackage{hyperref}
\hypersetup{
    pdftitle={\thetitle},
    pdfauthor={\theauthor},
    pdfsubject={SRS for Arion Software}
    pdfkeywords={SRS, Software, Specification, Arion, Flashcards, Spaced Repetition},
    colorlinks=true,   % false: boxed links; true: colored links
    linkcolor=black,   % color of internal links
    citecolor=magenta, % color of links to bibliography
    filecolor=orange,  % color of file links
    urlcolor=cyan,     % color of external links
    linktoc=page       % only page is linked
}

\begin{document}

\begin{flushright}
    \rule{\linewidth}{5pt}
    \vskip 1cm
    \begin{bfseries}
        \Huge
        SOFTWARE REQUIREMENTS\\
        SPECIFICATION\\
        \smallspace
        for\\
        \smallspace
        Arion\\
        \bigspace

        \LARGE
        Version \version \\
        \smallspace
        Prepared by \theauthor\\
        \smallspace
        \thedate\\
    \end{bfseries}
\end{flushright}

\tableofcontents


\chapter{Introduction}

\section{Purpose}
This document details the planned implementation of the flashcard spaced repetition software Arion.
It describes the functional and non-functional requirements of the system, providing documentation for users
and the software author.
It establishes prerequisites for the software to function, and outlines the planned life cycle of the product.

\section{Document Conventions}
This document was created based on a Latex template for Software Requirement Specification documents that
respect the IEEE standard.

\section{Intended Audience and Reading Suggestions}
\begin{itemize}
    \item Language learners, who want to use Arion to study new vocabulary or remember previously seen words.
    \item Students, who want to memorize content for their classes.
    \item Programmers who are interested in furthering the software by developing it or
fixing bugs. 
\end{itemize}

\section{Project Scope}
Arion is a system used to memorize information. Users can create, manage, and study a list of flashcards.
Arion allows users to build their memory over time, allowing them to remember concepts over long periods of time.

\section{References}
% flush left because this text is not intended to fill the whole line, so 
% it throws a warning (ANNOYING!!)
\begin{flushleft} 
    GitHub page: \\
    \url{https://github.com/Some-Guy-2017/arion} \\
    LaTeX Software Requirements Specification template: \\
    \url{https://github.com/jpeisenbarth/SRS-Tex} \\
\end{flushleft}


\chapter{Overall Description}

\section{Product Perspective}
Arion was built to be small and simple. \href{https://apps.ankiweb.net/}{Anki} is one of the most popular desktop
spaced repetition systems, but infamously has an unintuitive design.
Arion removes unnecessary features and has a simple GUI, improving user experience and making modification by users
with programming experience easy.
Arion is an open source, standalone application made for a senior year high school final project.
It runs on Windows, macOS, and Linux.

\section{Product Functions}
File:
\begin{itemize}
    \item Load: Loads flashcards from the local database.
    \item Save: Saves flashcards to the local database.
\end{itemize}

Edit:
\begin{itemize}
    \item Browse: Displays the flashcards in a table.
    \begin{itemize}
        \item Delete: Delete a flashcard.
        \item Edit: Edit a flashcard.
    \end{itemize}
    \item Add: Adds a flashcard to the list.
\end{itemize}

View:
\begin{itemize}
    \item Study: Study the flashcards.
    \item Sort: Sort the flashcards.
\end{itemize}

Help:
\begin{itemize}
    \item Guide: Display a user guide explaining how to use the program.
    \item About: Summarizes the program's functionality.
\end{itemize}

Quit: Quits the program.

\section{User Classes and Characteristics}
\begin{itemize}
    \item Language learners, who want to use Arion to study new vocabulary or remember previously seen words.
    \item Students, who want to memorize content for their classes.
    \item Programmers who are interested in furthering the software by developing it or fixing bugs.
\end{itemize}

\section{Operating Environment}

\begin{flushleft} % text does not fill the line, so must flush left to avoid error.
    Arion was designed to work on PCs with a Java Virtual Machine. \\

    It works on the following operating systems: \\
    \begin{itemize}
        \item Windows Vista
        \item Windows 7
        \item Windows 8
        \item Windows 10
        \item Windows 11
        \item Mac OS
        \item Linux
    \end{itemize}
\end{flushleft}

\section{Design and Implementation Constraints}
Arion is implemented in Java, using Java SWING and Abstract Took Kit (AWT) for the Graphical User Interface (GUI).
It reads and writes to the user’s machine, and requires the Java Virtual Machine (JVM) to be installed.

\section{User Documentation}
\begin{flushleft}
    A user guide is included in the Arion help menu. \\
    Arion has a user manual: \\
    \url{https://drive.google.com/file/d/1R3L0F9FErr2sn22dniuIB0acMaw8OTUr/view?usp=sharing}
\end{flushleft}

\section{Assumptions and Dependencies}
Arion is written in Java, so requires a JVM version of at least 17 to run.


\chapter{External Interface Requirements}

\section{User Interfaces}
\begin{itemize}
    \item
        Main Screen: The default screen that shows when the user first starts the program.
        It includes a study button that allows the user to enter study mode,
        and an add button that enters add mode.
    \item 
        Browse: A table listing the flashcards, with editable fields and allowing users to 
        delete entries. Back button allows users to return to the main screen.
    \item
        Study: Displays the flashcards and a bar with buttons. The bar either contains a 
        flip button to flip the card, or a correct button and an incorrect button that
        allows users to self judge their memory. Has a back button to return to 
        the main screen.
    \item 
        Sort: Has a drop-down allowing users to sort based on flashcard fronts or backs,
        and a drop-down for sorting in lexicographical or reverse lexicographical order. 
        Also has a button for confirming the sort, and a back button to return to the main screen.
    \item 
        Guide: A pop-up window with a number of screens, each detailing the usage of a
        different part of the program. Has an exit button to return to the main program.
    \item 
        About: A pop-up window detailing the Arion program, including its purpose and development.
        Has an exit button to return to the main program.
    \item 
        Add: A screen with two labeled editable text fields for users to enter the front and
        back of a new card. Also has a confirm button to add the flashcard to the list, and
        a back button to return.
\end{itemize}

\section{Hardware Interfaces}
Since Arion runs in a JVM, all hardware must be supported by Java.

\section{Software Interfaces}
Arion is written in Java, so requires a JVM version 17 or higher to be installed on the system.

\section{Communications Interfaces}
Arion is a standalone program, so has no external communications with other services.


\chapter{System Features}

\section{Load}
    \subsection{Stimulus/Response Sequence}
        \begin{flushleft}
            \makebox[1.7cm][l]{Stimulus:} User clicks the Load menu item. \\
            \makebox[1.7cm][l]{Stimulus:} User opens the program. \\
            \makebox[1.7cm][l]{Response:} \\
            \begin{enumerate}[1.]
                \item Ask the user to confirm overwriting the previous flashcards.
                \item Open the database file for reading.
                \item Parse the file for flashcards.
                \item Validate each flashcard.
                \item Write each flashcard into the local array.
                \item Close the file.
            \end{enumerate}
        \end{flushleft}

    \subsection{User Requirements}
        \begin{itemize}
            \item The user must open the program or click the Load menu item.
            \item The user must confirm overwriting their previous flashcards.
        \end{itemize}

    \subsection{System Requirements}
        \begin{itemize}
            \item The system must have permission to read from the database file.
            \item The system must have enough memory for the flashcards.
        \end{itemize}

\section{Save}
    \subsection{Stimulus/Response Sequence}
        \begin{flushleft}
            \makebox[1.7cm][l]{Stimulus:} User clicks the Save menu item. \\
            \makebox[1.7cm][l]{Stimulus:} User closes the program. \\
            \makebox[1.7cm][l]{Response:} \\
            \begin{enumerate}[1.]
                \item Ask the user to confirm overwriting the flashcards stored in the database file.
                \item Open the database file for writing.
                \item Write each flashcard to the file.
                \item Close the file.
            \end{enumerate}
        \end{flushleft}

    \subsection{User Requirements}
        \begin{itemize}
            \item The user must click the Save menu item or close the program.
            \item The user must confirm overwriting their previously stored flashcards.
        \end{itemize}

    \subsection{System Requirements}
        \begin{itemize}
            \item The system must have permission to write to the database file.
        \end{itemize}

\section{Browse}
    \subsection{Stimulus/Response Sequence}
        \begin{flushleft}
            \makebox[1.7cm][l]{Stimulus:} The user clicks the Browse menu item. \\
            \makebox[1.7cm][l]{Response:} The system brings up the Browse screen. \\
        \end{flushleft}

    \subsection{User Requirements}
        \begin{itemize}
            \item The user must click the Browse menu item.
        \end{itemize}

    \subsection{System Requirements}
        There are no system requirements.

\section{Delete}
    \subsection{Stimulus/Response Sequence}
        \begin{flushleft}
            \makebox[1.7cm][l]{Stimulus:} The user clicks the delete button next to the flashcard entry in the browse screen table. \\
            \makebox[1.7cm][l]{Response:} \\
            \begin{enumerate}[1.]
                \item Ask the user to confirm their selection.
                \item Delete the selected flashcard from the list.
                \item Update the displayed table.
            \end{enumerate}
        \end{flushleft}

    \subsection{User Requirements}
        \begin{itemize}
            \item The user must click the delete button in the Browse GUI.
            \item The user must confirm their selection.
        \end{itemize}

    \subsection{System Requirements}
        \begin{itemize}
            \item The system must be in browse mode.
        \end{itemize}

\section{Edit}
    \subsection{Stimulus/Response Sequence}
        \begin{flushleft}
            \makebox[1.7cm][l]{Stimulus:} The user edits a field in the flashcard table. \\
            \makebox[1.7cm][l]{Response:} The system stores the edited information. \\
            \smallspace
            \makebox[1.7cm][l]{Stimulus:} The user presses the Update button next to the flashcard in the browse screen table. \\
            \makebox[1.7cm][l]{Response:} The system writes the edited information to the internally stored list of flashcards. \\
        \end{flushleft}
    \subsection{User Requirements}
        \begin{itemize}
            \item The user must click the update button to update the information.
        \end{itemize}

    \subsection{System Requirements}
        \begin{itemize}
            \item The system must be in browse mode.
        \end{itemize}

\section{Add}
    \subsection{Stimulus/Response Sequence}
        \begin{flushleft}
            \makebox[1.7cm][l]{Stimulus:} The user clicks the Add menu item. \\
            \makebox[1.7cm][l]{Response:} The system brings up the add screen. \\
            \smallspace
            \makebox[1.7cm][l]{Stimulus:} The user clicks the Add button in the add screen. \\
            \makebox[1.7cm][l]{Response:} The system adds the new contact to the contact list. \\
        \end{flushleft}

    \subsection{User Requirements}
        \begin{itemize}
            \item The user must enter the new information for the contact.
            \item The user must click the Add button to confirm adding the new contact.
        \end{itemize}

    \subsection{System Requirements}
        \begin{itemize}
            \item The system must have enough memory for the new flashcard.
        \end{itemize}

\section{Study}
    \subsection{Stimulus/Response Sequence}
        \begin{flushleft}
            \makebox[1.7cm][l]{Stimulus:} The user clicks the study button  \\
            \makebox[1.7cm][l]{Response:} \\
            \begin{enumerate}[1.]
                \item The system brings up the study GUI.
                \item For each flashcard in the list of cards to study today:
                \begin{enumerate}[a.]
                    \item The system shows the front of the card.
                    \item The user flips the card
                    \item The user self-scores their performance.
                    \item The system adds the appropriate amount of time to the time until the card should be next seen.
                    \item The system puts the card into the 'today' list of cards, or the 'later' list.
                \end{enumerate}
            \end{enumerate}
        \end{flushleft}

    \subsection{User Requirements}
        \begin{itemize}
            \item The user must click the study button.
            \item The user must flip the flashcards.
            \item The user must self-score their knowledge of their memory.
        \end{itemize}

    \subsection{System Requirements}

\section{Sort}
    \subsection{Stimulus/Response Sequence}
        \begin{flushleft}
            \makebox[1.7cm][l]{Stimulus:} The user clicks the Sort menu item. \\
            \makebox[1.7cm][l]{Response:} The system brings up the sort GUI. \\
            \smallspace
            \makebox[1.7cm][l]{Stimulus:} The user clicks the Sort button in the sort GUI. \\
            \makebox[1.7cm][l]{Response:} The system sorts the flashcards based on the user's options. \\
        \end{flushleft}
    \subsection{User Requirements}
        \begin{itemize}
            \item The user must click the Sort menu item.
            \item The user must enter whether to sort the fronts or backs of the cards.
            \item The user must enter whether to sort the cards in lexicographical or reverse lexicographical order.
            \item The user must click the Sort button.
        \end{itemize} 

    \subsection{System Requirements}
        There are so system requirements. 

\section{Guide}
    \subsection{Stimulus/Response Sequence}
        \begin{flushleft}
            \makebox[1.7cm][l]{Stimulus:} The user clicks the Guide menu item. \\
            \makebox[1.7cm][l]{Response:} The system brings up the guide pop-up window. \\
        \end{flushleft}

    \subsection{User Requirements}
        \begin{itemize}
            \item The user must click the Guide menu item.
        \end{itemize}

    \subsection{System Requirements}
        There are no system requirements.

\section{About}
    \subsection{Stimulus/Response Sequence}
        \begin{flushleft}
            \makebox[1.7cm][l]{Stimulus:} The user clicks the About menu item. \\
            \makebox[1.7cm][l]{Response:} The system brings up the about pop-up window. \\
        \end{flushleft}

    \subsection{User Requirements}
        \begin{itemize}
            \item The user must click the About menu item.
        \end{itemize}

    \subsection{System Requirements}
        There are no system requirements.

\section{Quit}
    \subsection{Stimulus/Response Sequence}
        \begin{flushleft}
            \makebox[1.7cm][l]{Stimulus:} The user clicks the Quit menu item. \\
            \makebox[1.7cm][l]{Stimulus:} The user clicks the window exit button. \\
            \makebox[1.7cm][l]{Response:} \\
            \begin{enumerate}[1.]
                \item The system saves the flashcards to the database.
                \item The system closes the GUI.
                \item The system closes.
            \end{enumerate}
        \end{flushleft}

    \subsection{User Requirements}
        \begin{itemize}
            \item The user must click the Quit menu item or the window exit button.
        \end{itemize}

    \subsection{System Requirements}
        There are no system requirements.


\chapter{Milestones}
    \begin{enumerate}[1.]
        \item Create a detailed Software Requirements Specification.
        \item Create class and sequence diagrams detailing the project.
        \item Create a minimum viable product.
        \item Add remaining non-functional requirements.
        \item Perform unit and system testing.
        \item Create a user manual.
        \item Deploy the program.
    \end{enumerate}


\chapter{Other Requirements}

\section{Performance Requirements}
$<$If there are performance requirements for the product under various 
circumstances, state them here and explain their rationale, to help the 
developers understand the intent and make suitable design choices. Specify the 
timing relationships for real time systems. Make such requirements as specific 
as possible. You may need to state performance requirements for individual 
functional requirements or features.$>$

\section{Safety Requirements}
$<$Specify those requirements that are concerned with possible loss, damage, or 
harm that could result from the use of the product. Define any safeguards or 
actions that must be taken, as well as actions that must be prevented. Refer to 
any external policies or regulations that state safety issues that affect the 
product’s design or use. Define any safety certifications that must be 
satisfied.$>$

\section{Security Requirements}
$<$Specify any requirements regarding security or privacy issues surrounding use 
of the product or protection of the data used or created by the product. Define 
any user identity authentication requirements. Refer to any external policies or 
regulations containing security issues that affect the product. Define any 
security or privacy certifications that must be satisfied.$>$

\section{Software Quality Attributes}
$<$Specify any additional quality characteristics for the product that will be 
important to either the customers or the developers. Some to consider are: 
adaptability, availability, correctness, flexibility, interoperability, 
maintainability, portability, reliability, reusability, robustness, testability, 
and usability. Write these to be specific, quantitative, and verifiable when 
possible. At the least, clarify the relative preferences for various attributes, 
such as ease of use over ease of learning.$>$

\section{Business Rules}
$<$List any operating principles about the product, such as which individuals or 
roles can perform which functions under specific circumstances. These are not 
functional requirements in themselves, but they may imply certain functional 
requirements to enforce the rules.$>$


\chapter{Other Requirements}
$<$Define any other requirements not covered elsewhere in the SRS. This might 
include database requirements, internationalization requirements, legal 
requirements, reuse objectives for the project, and so on. Add any new sections 
that are pertinent to the project.$>$

\section{Appendix A: Glossary}
%see https://en.wikibooks.org/wiki/LaTeX/Glossary
$<$Define all the terms necessary to properly interpret the SRS, including 
acronyms and abbreviations. You may wish to build a separate glossary that spans 
multiple projects or the entire organization, and just include terms specific to 
a single project in each SRS.$>$

\section{Appendix B: Analysis Models}
$<$Optionally, include any pertinent analysis models, such as data flow 
diagrams, class diagrams, state-transition diagrams, or entity-relationship 
diagrams.$>$

\section{Appendix C: To Be Determined List}
$<$Collect a numbered list of the TBD (to be determined) references that remain 
in the SRS so they can be tracked to closure.$>$

\end{document}
