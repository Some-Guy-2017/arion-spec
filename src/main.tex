% Based on the template from Jean-Philippe Eisenbarth
% https://github.com/jpeisenbarth/SRS-Tex

\documentclass{scrreprt}
\title{Arion Software Requirements Specification}
\author{Joseph Hare}
\date{\today}

\setlength\parindent{0pt}

\makeatletter
\newcommand{\thetitle}{\@title}
\newcommand{\theauthor}{\@author}
\newcommand{\thedate}{\@date}
\makeatother

\newcommand{\version}{1.0 }
\newcommand{\bigspace}{\vspace{1.9cm}}
\newcommand{\smallspace}{\vspace{0.5cm}}

\newcommand{\image}[2][1]{
    \begin{center}
        \includegraphics[scale=#1]{#2}
    \end{center}
}

\usepackage{xcolor}
\usepackage{hyperref}
\usepackage[shortlabels]{enumitem}
\usepackage[super]{nth}

\usepackage{graphicx}
\graphicspath{ {../img/} }

\usepackage{hyperref}
\hypersetup{
    pdftitle={\thetitle},
    pdfauthor={\theauthor},
    pdfsubject={SRS for Arion Software}
    pdfkeywords={SRS, Software, Specification, Arion, Flashcards, Spaced Repetition},
    colorlinks=true,   % false: boxed links; true: colored links
    linkcolor=black,   % color of internal links
    citecolor=magenta, % color of links to bibliography
    filecolor=orange,  % color of file links
    urlcolor=cyan,     % color of external links
    linktoc=page       % only page is linked
}

\begin{document}

\begin{flushright}
    \rule{\linewidth}{5pt}
    \vskip 1cm
    \begin{bfseries}
        \Huge
        SOFTWARE REQUIREMENTS\\
        SPECIFICATION\\
        \smallspace
        for\\
        \smallspace
        Arion\\
        \bigspace

        \LARGE
        Version \version \\
        \smallspace
        Prepared by \theauthor\\
        \smallspace
        \thedate\\
    \end{bfseries}
\end{flushright}

\tableofcontents


\chapter{Introduction}

\section{Purpose}
This document details the planned implementation of the flashcard spaced repetition software Arion.
It describes the functional and non-functional requirements of the system, providing documentation for users
and the software author.
It establishes prerequisites for the software to function, and outlines the planned life cycle of the product.

\section{Document Conventions}
This document was created based on a Latex template for Software Requirement Specification documents that
respect the IEEE standard.

\section{Intended Audience and Reading Suggestions}
\begin{itemize}
    \item Language learners, who want to use Arion to study new vocabulary or remember previously seen words.
    \item Students, who want to memorize content for their classes.
    \item Programmers who are interested in furthering the software by developing it or
fixing bugs. 
\end{itemize}

\section{Project Scope}
Arion is a system used to memorize information. Users can create, manage, and study a list of flashcards.
Arion allows users to build their memory over time, allowing them to remember concepts over long periods of time.

\section{References}
% flush left because this text is not intended to fill the whole line, so 
% it throws a warning (ANNOYING!!)
\begin{flushleft} 
    GitHub page: \\
    \url{https://github.com/Some-Guy-2017/arion} \\
    LaTeX Software Requirements Specification template: \\
    \url{https://github.com/jpeisenbarth/SRS-Tex} \\
    Spaced Repetition History: \\
    \url{https://files.eric.ed.gov/fulltext/EJ1143520.pdf} \\
    Wikipedia page for the Leitner system: \\
    \url{https://en.wikipedia.org/wiki/Leitner_system} \\
\end{flushleft}


\chapter{Overall Description}

\section{Product Perspective}
Arion was built to be small and simple.
Arion has no unnecessary features and a simple GUI, improving user experience and making modification by users
with programming experience easy.
Arion is an open source, standalone application made for a senior year high school final project.
It runs on Windows, macOS, and Linux.

\section{Product Functions}
File:
\begin{itemize}
    \item Load: Loads flashcards from the local database.
    \item Save: Saves flashcards to the local database.
\end{itemize}

Edit:
\begin{itemize}
    \item Browse: Displays the flashcards in a table.
    \begin{itemize}
        \item Delete: Delete a flashcard.
        \item Edit: Edit a flashcard.
    \end{itemize}
    \item Add: Adds a flashcard to the list.
\end{itemize}

View:
\begin{itemize}
    \item Study: Study the flashcards.
    \item Sort: Sort the flashcards.
\end{itemize}

Help:
\begin{itemize}
    \item Guide: Display a user guide explaining how to use Arion.
    \item About: Summarizes Arion's functionality.
\end{itemize}

Quit: Quits Arion.

\section{User Classes and Characteristics}
\begin{itemize}
    \item Language learners, who want to use Arion to study new vocabulary or remember previously seen words.
    \item Students, who want to memorize content for their classes.
    \item Programmers who are interested in furthering the software by developing it or fixing bugs.
\end{itemize}

\section{Operating Environment}

\begin{flushleft} % text does not fill the line, so must flush left to avoid error.
    Arion was designed to work on PCs with a Java Virtual Machine. \\

    It works on the following operating systems: \\
    \begin{itemize}
        \item Windows Vista
        \item Windows 7
        \item Windows 8
        \item Windows 10
        \item Windows 11
        \item Mac OS
        \item Linux
    \end{itemize}
\end{flushleft}

\section{Design and Implementation Constraints}
Arion is implemented in Java, using Java SWING and Abstract Took Kit (AWT) for the Graphical User Interface (GUI).
It reads and writes to the user’s machine, and requires the Java Virtual Machine (JVM) to be installed.

\section{User Documentation}
\begin{flushleft}
    A user guide is included in the Arion help menu. \\
    Arion has a user manual: \\
    \url{https://drive.google.com/file/d/1R3L0F9FErr2sn22dniuIB0acMaw8OTUr/view?usp=sharing}
\end{flushleft}

\section{Assumptions and Dependencies}
Arion is written in Java, so requires a JVM version of at least 17 to run.


\chapter{External Interface Requirements}

\section{User Interfaces}
\begin{itemize}
    \item
        Main Screen: The default screen that shows when the user first starts Arion.
        It includes a study button that allows the user to enter study mode,
        and an add button that enters add mode.
    \item 
        Browse: A table listing the flashcards, with editable fields and allowing users to 
        delete entries. Back button allows users to return to the main screen.
    \item
        Study: Displays the flashcards and a bar with buttons. The bar either contains a 
        flip button to flip the card, or a correct button and an incorrect button that
        allows users to self judge their memory. Has a back button to return to 
        the main screen.
    \item 
        Sort: Has a drop-down allowing users to sort based on flashcard fronts or backs,
        and a drop-down for sorting in lexicographical or reverse lexicographical order. 
        Also has a button for confirming the sort, and a back button to return to the main screen.
    \item 
        Guide: A pop-up window with a number of screens, each detailing the usage of a
        different part of Arion. Has an exit button to return to the main program.
    \item 
        About: A pop-up window detailing the Arion program, including its purpose and development.
        Has an exit button to return to the main program.
    \item 
        Add: A screen with two labeled editable text fields for users to enter the front and
        back of a new card. Also has a confirm button to add the flashcard to the list, and
        a back button to return.
\end{itemize}

\section{Hardware Interfaces}
Since Arion runs in a JVM, all hardware must be supported by Java.

\section{Software Interfaces}
Arion is written in Java, so requires a JVM version 18 or higher to be installed on the system.

\section{Communications Interfaces}
Arion is a standalone program, so has no external communications with other services.


\chapter{System Features}

\section{Load}
    \subsection*{Stimulus/Response Sequence}
        \begin{flushleft}
            \makebox[1.7cm][l]{Stimulus:} User clicks the Load menu item. \\
            \makebox[1.7cm][l]{Stimulus:} User opens Arion. \\
            \makebox[1.7cm][l]{Response:} The system loads the flashcards from the database file. \\
        \end{flushleft}

    \subsection*{User Requirements}
        \begin{itemize}
            \item The user must open Arion or click the Load menu item.
            \item The user must confirm overwriting their previous flashcards.
        \end{itemize}

    \subsection*{System Requirements}
        \begin{itemize}
            \item The system must have permission to read from the database file.
            \item The system must have enough memory for the flashcards.
        \end{itemize}

\section{Save}
    \subsection*{Stimulus/Response Sequence}
        \begin{flushleft}
            \makebox[1.7cm][l]{Stimulus:} User clicks the Save menu item. \\
            \makebox[1.7cm][l]{Stimulus:} User closes Arion. \\
            \makebox[1.7cm][l]{Response:} Save the program flashcards to the database file.
        \end{flushleft}

    \subsection*{User Requirements}
        \begin{itemize}
            \item The user must click the Save menu item or close Arion.
            \item The user must confirm overwriting their previously stored flashcards.
        \end{itemize}

    \subsection*{System Requirements}
        \begin{itemize}
            \item The system must have permission to write to the database file.
        \end{itemize}

\section{Browse}
    \subsection*{Stimulus/Response Sequence}
        \begin{flushleft}
            \makebox[1.7cm][l]{Stimulus:} The user clicks the Browse menu item. \\
            \makebox[1.7cm][l]{Response:} The system brings up the Browse screen. \\
        \end{flushleft}

    \subsection*{User Requirements}
        \begin{itemize}
            \item The user must click the Browse menu item.
        \end{itemize}

    \subsection*{System Requirements}
        \begin{itemize}
            \item There must be flashcards to display.
        \end{itemize}

\section{Delete}
    \subsection*{Stimulus/Response Sequence}
        \begin{flushleft}
            \makebox[1.7cm][l]{Stimulus:} The user clicks the delete button next to the flashcard entry in the browse screen table. \\
            \makebox[1.7cm][l]{Response:} Delete the selected flashcard. \\
        \end{flushleft}

    \subsection*{User Requirements}
        \begin{itemize}
            \item The user must click the delete button in the Browse GUI.
            \item The user must confirm their selection.
        \end{itemize}

    \subsection*{System Requirements}
        \begin{itemize}
            \item The system must be in browse mode.
        \end{itemize}

\section{Edit}
    \subsection*{Stimulus/Response Sequence}
        \begin{flushleft}
            \makebox[1.7cm][l]{Stimulus:} The user edits a field in the flashcard table. \\
            \makebox[1.7cm][l]{Response:} The system stores the edited information. \\
            \smallspace
            \makebox[1.7cm][l]{Stimulus:} The user presses the Update button next to the flashcard in the browse screen table. \\
            \makebox[1.7cm][l]{Response:} The system writes the edited information to the internally stored list of flashcards. \\
        \end{flushleft}
    \subsection*{User Requirements}
        \begin{itemize}
            \item The user must click the update button to update the information.
        \end{itemize}

    \subsection*{System Requirements}
        \begin{itemize}
            \item The system must be in browse mode.
        \end{itemize}

\section{Add}
    \subsection*{Stimulus/Response Sequence}
        \begin{flushleft}
            \makebox[1.7cm][l]{Stimulus:} The user clicks the Add menu item. \\
            \makebox[1.7cm][l]{Response:} The system brings up the add screen. \\
            \smallspace
            \makebox[1.7cm][l]{Stimulus:} The user clicks the Add button in the add screen. \\
            \makebox[1.7cm][l]{Response:} The system adds the new contact to the contact list. \\
        \end{flushleft}

    \subsection*{User Requirements}
        \begin{itemize}
            \item The user must enter the new information for the contact.
            \item The user must click the Add button to confirm adding the new contact.
        \end{itemize}

    \subsection*{System Requirements}
        \begin{itemize}
            \item The system must have enough memory for the new flashcard.
        \end{itemize}

\section{Study}
    \subsection*{Stimulus/Response Sequence}
        \begin{flushleft}
            \makebox[1.7cm][l]{Stimulus:} The user clicks the study button  \\
            \makebox[1.7cm][l]{Response:} Arion enters study mode. \\
        \end{flushleft}

    \subsection*{User Requirements}
        \begin{itemize}
            \item The user must click the study button.
            \item The user must flip the flashcards.
            \item The user must self-score their knowledge of their memory.
        \end{itemize}

    \subsection*{System Requirements}

\section{Sort}
    \subsection*{Stimulus/Response Sequence}
        \begin{flushleft}
            \makebox[1.7cm][l]{Stimulus:} The user clicks the Sort menu item. \\
            \makebox[1.7cm][l]{Response:} The system brings up the sort GUI. \\
            \smallspace
            \makebox[1.7cm][l]{Stimulus:} The user clicks the Sort button in the sort GUI. \\
            \makebox[1.7cm][l]{Response:} The system sorts the flashcards based on the user's options. \\
        \end{flushleft}
    \subsection*{User Requirements}
        \begin{itemize}
            \item The user must click the Sort menu item.
            \item The user must enter whether to sort the fronts or backs of the cards.
            \item The user must enter whether to sort the cards in lexicographical or reverse lexicographical order.
            \item The user must click the Sort button.
        \end{itemize} 

    \subsection*{System Requirements}
        There are so system requirements. 

\section{Guide}
    \subsection*{Stimulus/Response Sequence}
        \begin{flushleft}
            \makebox[1.7cm][l]{Stimulus:} The user clicks the Guide menu item. \\
            \makebox[1.7cm][l]{Response:} The system brings up the guide pop-up window. \\
        \end{flushleft}

    \subsection*{User Requirements}
        \begin{itemize}
            \item The user must click the Guide menu item.
        \end{itemize}

    \subsection*{System Requirements}
        There are no system requirements.

\section{About}
    \subsection*{Stimulus/Response Sequence}
        \begin{flushleft}
            \makebox[1.7cm][l]{Stimulus:} The user clicks the About menu item. \\
            \makebox[1.7cm][l]{Response:} The system brings up the about pop-up window. \\
        \end{flushleft}

    \subsection*{User Requirements}
        \begin{itemize}
            \item The user must click the About menu item.
        \end{itemize}

    \subsection*{System Requirements}
        There are no system requirements.

\section{Quit}
    \subsection*{Stimulus/Response Sequence}
        \begin{flushleft}
            \makebox[1.7cm][l]{Stimulus:} The user clicks the Quit menu item. \\
            \makebox[1.7cm][l]{Stimulus:} The user clicks the window exit button. \\
            \makebox[1.7cm][l]{Response:} Arion quits. \\
        \end{flushleft}

    \subsection*{User Requirements}
        \begin{itemize}
            \item The user must click the Quit menu item or the window exit button.
        \end{itemize}

    \subsection*{System Requirements}
        There are no system requirements.


\chapter{Milestones}
    \begin{enumerate}[1.]
        \item Create a detailed Software Requirements Specification.
        \item Create class and sequence diagrams detailing the project.
        \item Create a minimum viable product.
        \item Add remaining non-functional requirements.
        \item Perform unit and system testing.
        \item Deploy Arion.
    \end{enumerate}


\chapter{Key Resource Requirements}

\section{Major tasks}
Since this is being developed by a single developer, all tasks are assigned to Joseph Hare.
\begin{itemize}
    \item Requirement documentation.
    \item Class and sequence diagram creation.
    \item GUI implementation.
    \item Back end implementation.
    \item Testing.
    \item User manual creation.
\end{itemize}


\chapter{Other Requirements}

\section{Performance Requirements}
\begin{itemize}
    \item GUIs shall load within 10ms, otherwise Arion will feel unresponsive to users.
    \item Sorting shall be performed within 20ms for 1,000 flashcards.
    \item Adding, editing, and deleting flashcards shall take no more than 5ms.
    \item Displaying flashcards shall take no more than 5ms.
        Since this process occurs frequently, a high delay would irritate users.
\end{itemize}

\section{Software Quality Attributes}
\begin{itemize}
    \item Arion shall be modular and written in an object-oriented programming language,
        to allow for future updates and ease of creation.
    \item Arion shall be programmed in a portable programming language,
        such as Java, to allow for future porting to other systems.
    \item Arion shall be reliable and robust, so no unexpected inputs will
        make it fail.
\end{itemize}


\chapter{Appendices}

\section{Glossary}
\begin{itemize}
    \item Flashcard: A card containing a prompt on the front and information on the back,
        which is used to memorize information.
    \item Spaced repetition: a way to memorize information. New and difficult information
        is show more frequently than easy and older information, mimicking the pattern
        by which people forget information to optimize learning rate.
    \item Java: A multi-platform object-oriented programming language.
    \item Multi-platform: Able to run on many platforms without needing to alter the code.
    \item Object-oriented: a style of programming where data and code are organized into
        separate "objects."
    \item Graphical User Interface (GUI): A visible digital interface where users can 
        interact with the program using buttons, icons, etc.
    \item Java Abstract Window Toolkit (AWT): Java's first graphics toolkit, allowing developers
        to provide a GUI for their programs.
    \item Java SWING: A newer graphical toolkit for Java that can emulate the look of multiple platforms.
    \item Virtual Machine (VM): A computer implemented in software, as opposed to hardware.
    \item Java Virtual Machine (JVM): Java's VM used to run compiled Java programs.
\end{itemize}

\section{Analysis Models}
\image[0.4]{arion-use-case-diagram}

\section{Project Proposal}

\subsection{Problem Statement}
Traditional study methods often result in forgetting information over long periods of time.
Memorization over years, or even a lifetime, is crucial for
speaking a second language, entering a new profession, and many more scenarios.
A standardized approach to learning information is thus needed.

\subsection{Objectives}
Arion primarily aims to provide a spaced-repetition system to allow learners to 
systemically built their knowledge over long periods of time. \\
Arion's objectives are to:
\begin{itemize}
    \item Provide a friendly user interface for managing flashcards.
    \item Be open-source to allow free usage of Arion.
    \item Work quickly and robustly for excellent user experience.
    \item Allow future cross-platform availability.
    \item Written in a modular way for future expansions.
    \item Written clearly using DRY principals for easy development and updates.
\end{itemize}

\subsection{Methodology}
The development of Arion will following process:
\begin{enumerate}[1.]
    \item Create a detailed specification for the development process to follow.
    \item Detail the specifics of Arion using class and sequence diagrams.
    \item Implement the back-end of Arion (card creation, deletion, editing, etc.).
    \item Implement the front-end of Arion (the user interface, buttons, labels, etc.).
    \item Rigorously test the product using both unit and system tests.
    \item Create a user manual.
    \item Deploy Arion.
    \item Iteratively improving Arion with new features.
\end{enumerate}

\subsection{History of Product}
Spaced repetition started in the \nth{19} century, when in 1885 Hermann Ebbinghaus
hypothesized that the rate at which people forget information increases exponentially
over time. Further study yielded the following two principals:
\begin{enumerate}[1.]
    \item Remembering the information leads to better memory than being shown the item.
    \item Remembering the information after a delay increases memory retention compared
        to recall soon after learning the item.
\end{enumerate}
The optimal strategy is to have learners review the information just before it will
be forgotten. Several estimations of when people will forget knowledge have been made;
Arion uses the Leitner system to temporally space flashcards.

\addtocontents{toc}{\protect\setcounter{tocdepth}{-1}} % hide chapters from ToC
\setcounter{chapter}{0} % reset numbering

\chapter{Arion - Load}

\section{Description}
Load loads the flashcards from the database into the program.

\section{Actors}
\begin{itemize}
    \item User
    \item Database
\end{itemize}

\section{Preconditions}
\begin{itemize}
    \item There is a database file for Arion to read from.
\end{itemize}

\section{Expected Flow of Event}
\begin{enumerate}[1.]
    \item The user clicks the Load button.
    \item Ask the user to confirm overwriting the previous flashcards.
    \item Open the database file for reading.
    \item Parse the file for flashcards.
    \item Validate each flashcard.
    \item Write each flashcard into a local array.
    \item Close the file.
\end{enumerate}

\section{Alternative Flow of Event}

    \subsection{No File}
    If in step 3 there is no file to read:
    \begin{enumerate}
        \item Print an error message to the user.
        \item End the use case with a failure condition.
    \end{enumerate}

    \subsection{Invalid Flashcard}
    If in step 5 a flashcard is incorrect:
    \begin{enumerate}
        \item Print a message stating that the file is formatted incorrectly.
        \item End the use case with a failure condition.
    \end{enumerate}

\section{Key Scenarios}
    \subsection{Load Flashcards}
    The flashcards are loaded from the database file.

    \subsection{Failure Loading}
    Arion exits without loading.

\section{Post-Conditions}
    \subsection{Successful Condition}
    The flashcards were loaded into memory.

    \subsection{Failure Condition}
    A failure message was printed.

\chapter{Arion - Save}

\section{Description}
Save saves the flashcards loaded in memory to the database file.

\section{Actors}
\begin{itemize}
    \item User
    \item Database
\end{itemize}

\section{Preconditions}
\begin{itemize}
    \item The system must have permission to write to the database file.
\end{itemize}

\section{Expected Flow of Event}
\begin{enumerate}[1.]
    \item Ask the user to confirm overwriting the flashcards stored in the database file.
    \item Open the database file for writing.
    \item Write each flashcard to the file.
    \item Close the file.
\end{enumerate}

\section{Alternative Flow of Event}

    \subsection{User Declines}
    If in step 1 the user declines to overwrite the flashcards,
    end the use case with a failure condition.

    \subsection{File Failure}
    If in step 2 the file cannot be opened for writing:
    \begin{enumerate}[1.]
        \item Print an error message.
        \item End the use case with a failure condition.
    \end{enumerate}

\section{Key Scenarios}
    \subsection{Save Flashcards}
    The flashcards are saved to the database file.
    
    \subsection{User Declines}
    The user declines to overwrite the existing flashcards.

    \subsection{Permission Error}
    Arion does not have permission to write to the database file.

\section{Post-Conditions}
    \subsection{Successful Condition}
    The flashcards were saved to the database file.
    
    \subsection{Failure Condition}
    The database file is unchanged, and an error message was printed.

\chapter{Arion - Browse}

\section{Description}
Browse displays the flashcards in a table for the user to edit or delete.

\section{Actors}
\begin{itemize}
    \item User
\end{itemize}

\section{Preconditions}
\begin{itemize}
    \item There must be flashcards to display.
\end{itemize}

\section{Expected Flow of Event}
\begin{enumerate}[1.]
    \item The user clicks the Browse button.
    \item Check that there are flashcards to display.
    \item Construct a GUI table.
    \item Fill the table with the flashcard fields.
    \item Display the table.
\end{enumerate}

\section{Alternative Flow of Event}

    \subsection{No Flashcards}
    If in step 2 there are no flashcards:
    \begin{enumerate}
        \item Display to the user that there are no flashcards.
        \item End the use case with a failure condition.
    \end{enumerate}

\section{Key Scenarios}
    \subsection{Display the Flashcards}
    The flashcards are displayed to the user in the form of a table.

    \subsection{Display No Flashcard Message}
    In the circumstance that there are no flashcards to display, Arion will
    display this to the user.

\section{Post-Conditions}
    \subsection{Successful Condition}
    The flashcards were displayed.
    
    \subsection{Failure Condition}
    The program displayed that there are no flashcards to display.


\chapter{Arion - Delete}

\section{Description}
Delete removes a flashcard from the list stored in memory.

\section{Actors}
\begin{itemize}
    \item User
\end{itemize}

\section{Preconditions}
\begin{itemize}
    \item Arion is in browse mode.
\end{itemize}

\section{Expected Flow of Event}
\begin{enumerate}[1.]
    \item Ask the user to confirm their selection.
    \item Delete the selected flashcard from the list.
    \item Update the displayed table.
\end{enumerate}

\section{Alternative Flow of Event}

    \subsection{User Declines}
    If in step 1 the user declines the selection,
    end the use case with a success condition.

\section{Key Scenarios}
    \subsection{KEYSCENARIO}

\section{Post-Conditions}
    \subsection{Successful Condition}
    
    \subsection{Failure Condition}


\chapter{Arion - Edit}

\section{Description}

\section{Actors}
\begin{itemize}
    \item 
    \item 
\end{itemize}

\section{Preconditions}
\begin{itemize}
    \item 
\end{itemize}

\section{Expected Flow of Event}
\begin{enumerate}[1.]
    \item 
\end{enumerate}

\section{Alternative Flow of Event}

    \subsection{ALTFLOW}
    If in step n ...:
    \begin{enumerate}
        \item 
    \end{enumerate}

\section{Key Scenarios}
    \subsection{KEYSCENARIO}

\section{Post-Conditions}
    \subsection{Successful Condition}
    
    \subsection{Failure Condition}


\chapter{Arion - Add}

\section{Description}

\section{Actors}
\begin{itemize}
    \item 
    \item 
\end{itemize}

\section{Preconditions}
\begin{itemize}
    \item 
\end{itemize}

\section{Expected Flow of Event}
\begin{enumerate}[1.]
    \item 
\end{enumerate}

\section{Alternative Flow of Event}

    \subsection{ALTFLOW}
    If in step n ...:
    \begin{enumerate}
        \item 
    \end{enumerate}

\section{Key Scenarios}
    \subsection{KEYSCENARIO}

\section{Post-Conditions}
    \subsection{Successful Condition}
    
    \subsection{Failure Condition}


\chapter{Arion - Study}

\section{Description}

\section{Actors}
\begin{itemize}
    \item 
    \item 
\end{itemize}

\section{Preconditions}
\begin{itemize}
    \item 
\end{itemize}

\section{Expected Flow of Event}
\begin{enumerate}[1.]
    \item The system brings up the study GUI.
    \item For each flashcard in the list of cards to study today:
    \begin{enumerate}[a.]
        \item The system shows the front of the card.
        \item The user flips the card
        \item The user self-scores their performance.
        \item The system adds the appropriate amount of time to the time until the card should be next seen.
        \item The system puts the card into the 'today' list of cards, or the 'later' list.
    \end{enumerate}
\end{enumerate}

\section{Alternative Flow of Event}
There are no alternative flows of events.

\section{Key Scenarios}
    \subsection{KEYSCENARIO}

\section{Post-Conditions}
    \subsection{Successful Condition}
    
    \subsection{Failure Condition}


\chapter{Arion - Sort}

\section{Description}

\section{Actors}
\begin{itemize}
    \item 
    \item 
\end{itemize}

\section{Preconditions}
\begin{itemize}
    \item 
\end{itemize}

\section{Expected Flow of Event}
\begin{enumerate}[1.]
    \item 
\end{enumerate}

\section{Alternative Flow of Event}

    \subsection{ALTFLOW}
    If in step n ...:
    \begin{enumerate}
        \item 
    \end{enumerate}

\section{Key Scenarios}
    \subsection{KEYSCENARIO}

\section{Post-Conditions}
    \subsection{Successful Condition}
    
    \subsection{Failure Condition}


\chapter{Arion - Guide}

\section{Description}

\section{Actors}
\begin{itemize}
    \item 
    \item 
\end{itemize}

\section{Preconditions}
\begin{itemize}
    \item 
\end{itemize}

\section{Expected Flow of Event}
\begin{enumerate}[1.]
    \item 
\end{enumerate}

\section{Alternative Flow of Event}

    \subsection{ALTFLOW}
    If in step n ...:
    \begin{enumerate}
        \item 
    \end{enumerate}

\section{Key Scenarios}
    \subsection{KEYSCENARIO}

\section{Post-Conditions}
    \subsection{Successful Condition}
    
    \subsection{Failure Condition}


\chapter{Arion - About}

\section{Description}

\section{Actors}
\begin{itemize}
    \item 
    \item 
\end{itemize}

\section{Preconditions}
\begin{itemize}
    \item 
\end{itemize}

\section{Expected Flow of Event}
\begin{enumerate}[1.]
    \item 
\end{enumerate}

\section{Alternative Flow of Event}

    \subsection{ALTFLOW}
    If in step n ...:
    \begin{enumerate}
        \item 
    \end{enumerate}

\section{Key Scenarios}
    \subsection{KEYSCENARIO}

\section{Post-Conditions}
    \subsection{Successful Condition}
    
    \subsection{Failure Condition}


\chapter{Arion - Quit}

\section{Description}

\section{Actors}
\begin{itemize}
    \item 
    \item 
\end{itemize}

\section{Preconditions}
\begin{itemize}
    \item 
\end{itemize}

\section{Expected Flow of Event}
    \begin{enumerate}[1.]
        \item The system saves the flashcards to the database.
        \item The system closes the GUI.
        \item The system closes.
    \end{enumerate}

\section{Alternative Flow of Event}
    There are no alternative flows of events.

\section{Key Scenarios}
    \subsection{KEYSCENARIO}

\section{Post-Conditions}
    \subsection{Successful Condition}
    
    \subsection{Failure Condition}


\end{document}
